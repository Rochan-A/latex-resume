%-------------------------
% Resume in Latex
% Author : Rochan Avlur Venkat
% License : MIT
%------------------------

\documentclass[a4paper,11pt]{article}

\usepackage[empty]{fullpage}
\usepackage{titlesec}
\usepackage[usenames,dvipsnames]{color}
\usepackage{verbatim}
\usepackage{enumitem}
\usepackage[pdftex]{hyperref}
\usepackage{fancyhdr}
\usepackage{kpfonts}
\usepackage{changepage}
\usepackage[T1]{fontenc}

\fancyhf{}
\renewcommand{\headrulewidth}{0pt}
\renewcommand{\footrulewidth}{0pt}

% Adjust margins
\addtolength{\oddsidemargin}{-0.3in}
\addtolength{\evensidemargin}{-0.3in}
\addtolength{\textwidth}{0.6in}
\addtolength{\topmargin}{-.3in}
\addtolength{\textheight}{0.5in}
\urlstyle{same}

\raggedright
\setlength{\tabcolsep}{0in}

% Sections formatting
\titleformat{\section}{
    \vspace{-4pt}\scshape\raggedright\large
}{}{0em}{}[\color{black}\titlerule \vspace{-5pt}]

%-------------------------
% Custom commands
\newcommand{\resumeItem}[2]{
    \item\small{
        \textbf{#1}{: #2 \vspace{-2pt}}
    }
}

\newcommand{\resumeItemSingle}[1]{
    \vspace{-2pt}\item\small{
        {#1}
    }
}

\newcommand{\resumeSubheading}[4]{
    \item
        \begin{tabular*}{0.97\textwidth}{l@{\extracolsep{\fill}}r}
            \textbf{#1} & #2 \\
            \small#3 & \textit{\small #4} \\
        \end{tabular*}\vspace{-5pt}
}

\newcommand{\resumeSubheadingResearch}[4]{
    \item
        \begin{tabular*}{0.97\textwidth}{l@{\extracolsep{\fill}}r}
            \textbf{#1} & \small #2 \\
            \textit{\small #3} & \textit{\small #4} \\
        \end{tabular*}\vspace{-5pt}
}

\newcommand{\resumeSubItem}[2]{\resumeItem{#1}{#2}\vspace{-4pt}}
\newcommand{\resumeSubHeadingListStart}{\begin{itemize}[leftmargin=*]}
\newcommand{\resumeSubHeadingListEnd}{\end{itemize}}
\newcommand{\resumeItemListStart}{\begin{itemize}[leftmargin=13pt]}
\newcommand{\resumeItemListEnd}{\end{itemize}\vspace{-5pt}}

%-------------------------------------------
%%%%%%  CV STARTS HERE  %%%%%%%%%%%%%%%%%%%%%%%%%%%%

\begin{document}

%----------HEADING-----------------
\begin{center}
\Large \textbf{Rochan Avlur Venkat}
\end{center}
\begin{tabular*}{\textwidth}{l@{\extracolsep{\fill}}r}
    \small \href{https://rochan-a.github.io}{rochan-a.github.io} & \small \href{mailto:rochan170543@mechyd.ac.in}{rochan170543@mechyd.ac.in}\\
    \small Phone: +91 91210 06945 & \small GitHub: \href{https://www.github.com/Rochan-A}{@Rochan-A}
\end{tabular*}

%-----------EDUCATION-----------------
\section{Education}
    \resumeSubHeadingListStart
        \item[] \begin{tabular*}{0.97\textwidth}{l@{\extracolsep{\fill}}r}
                \textbf{Mahindra \'{E}cole Centrale} & Hyderabad, India \\
                \small Bachelors of Technology in Computer Science \& Engineering & \textit{\small Jul. 2017 - May 2021\footnotemark[1]} \\
                \small GPA: 9.0/10 & \\
        \end{tabular*}
        \footnotetext[1]{Expected}
    \resumeSubHeadingListEnd

%-----------EXPERIENCE-----------------
\section{Research Experience}
    \resumeSubHeadingListStart
        \resumeSubheadingResearch
            {Research Intern}{University of Texas, Austin}
            {Mentor: \href{http://www.cs.utexas.edu/~bajaj/cvc/}{Prof. Chandrajit Bajaj}}{Jun. 2020 - present}
            \resumeItemListStart
                \resumeItemSingle{Goal is to develop a system to accurately and efficiently predict prognosis of cancer biopsies.}
                \resumeItemSingle{Uses Multi-Task Reinforcement Learning \& Variational Auto-Encoders to analyse large amounts of data efficiently by sampling a minimal amount of crucial information.}
            \resumeItemListEnd

        \resumeSubheadingResearch
            {Training Neural Networks with Evolutionary Optimization}{Mahindra \'{E}cole Centrale}
                {with \href{https://www.mahindraecolecentrale.edu.in/faculty/arya-K-bhattacharya}{Prof. Arya K Bhattacharya} \& Zakaria Oussalem}{Jan. 2020 - Jul. 2020}
                \resumeItemListStart
                    \resumeItemSingle{Studies the problem of training Neural Networks using Differential Evolution.}
                    \resumeItemSingle{Uses task apportioning algorithm that attains in certain cases $\times350$ speedup on hybrid CPU-GPU architectures.}
                \resumeItemListEnd
    
                \resumeSubheadingResearch
            {Feature extraction for Reinforcement Learning}{Mahindra \'{E}cole Centrale}
                {with \href{https://www.mahindraecolecentrale.edu.in/faculty/achal-agarwal}{Dr. Achal Agarwal}}{Jan. 2019 - Jan. 2020}
                \resumeItemListStart
                    \resumeItemSingle{Studies feature extraction techniques for Deep Reinforcement Learning in Atari 2600 environment.}
                    \resumeItemSingle{Motivated by \href{http://papers.nips.cc/paper/7811-unsupervised-video-object-segmentation-for-deep-reinforcement-learning}{\textbf{Goel \textit{et~al.}}} \& \href{https://arxiv.org/abs/1809.06064}{\textbf{Li \textit{et~al.}}} use of unsupervised image segmentation models \& computer vision techniques in combination with DQN (\& its variants), A2C and PPO.}
                \resumeItemListEnd

        \resumeSubheadingResearch
        {Summer Research Intern}{International Institute of Information Technology, Bangalore}
            {Mentor: \href{https://www.iiitb.ac.in/faculty/srinath-srinivasa}{Prof. Srinath Srinivasa} \& Chaitali Diwan}{May. 2019 - Jul. 2019}
            \resumeItemListStart
                \resumeItemSingle{Developed NLP models for generating \& validating semantic context and exposition coherence between learning resources in a learning pathway.}
                \resumeItemSingle{Developed a language model to generate virtual documents called \textbf{\href{https://github.com/Rochan-A/topic2document}{topic2document}} from topic distributions.}
            \resumeItemListEnd

        \resumeSubheadingResearch
        {Summer Intern}{ShowUpHotels, Bangalore}
            {Mentor: \href{https://in.linkedin.com/in/anupam-mediratta-60b5547}{Anupam Mediratta}}{May. 2018 - Jul. 2018}
            \resumeItemListStart
                \resumeItemSingle{Studied NLP (lda2vec, doc2vec \& word2vec) and topic models (LDA, LSA, NMF \& TFIDF).}
                \resumeItemSingle{Developed and published a Python package on PYPI called \textbf{\href{https://github.com/Rochan-A/sptm}{sptm}} (Sentence Prediction using Topic Modeling).}
            \resumeItemListEnd

    \resumeSubheadingResearch
        {Video Compression Algorithm}{Bangalore}
            {with \href{https://in.linkedin.com/in/chandra-vaidyanathan-14b22966}{Dr. Chandrashekar Vaidhyanathan}}{Aug. 2016 - May. 2017}
            \resumeItemListStart
                \resumeItemSingle{This work proposes a lossless compression scheme that can be extended to both Near-Lossless and Lossy compression of video data.}
                \resumeItemSingle{Extends on the work by \href{http://citeseerx.ist.psu.edu/viewdoc/download?doi=10.1.1.87.3641&rep=rep1&type=pdf}{\textbf{Davies \textit{et~al.}}} on using Bayesian Networks for lossless dataset compression.}
            \resumeItemListEnd
    \resumeSubHeadingListEnd

%-----------Papers & Publications-----------------
\section{Publications}
    \resumeSubHeadingListStart
        \item[] \textbf{Training Convolutional Neural Networks with Differential Evolution using Concurrent Task Apportioning on Hybrid CPU-GPU Architectures}
        \vspace*{-3pt}\begin{adjustwidth}{13pt}{}
            \textit{\underline{Rochan Avlur Venkat}, Zakaria Oussalem, Arya K Bhattacharya} \\ \textit{Under review}
        \end{adjustwidth}

        \item[] \textbf{Lossless Video Compression Using Bayesian Networks and Entropy Coding}
        \vspace*{-3pt}\begin{adjustwidth}{13pt}{}
            \textit{\underline{Rochan Avlur Venkat}, Chandrasekar Vaidyanathan} \\
            IEEE Region 10 Symposium (IEEE TENSYMP), Kolkata, India, 2019
        \end{adjustwidth}
    \resumeSubHeadingListEnd


%----------Projects----------------
\section{Selected Projects}
\resumeSubHeadingListStart
    \small
    \item{\textbf{Distributed Compute Fabric using Mobile Devices}{: Framework to exploit under-utilization of compute resources in smartphones}
    \vspace*{-8pt}\item{\textbf{AI/ML for Cart Conversion}{: Solution to cart-conversion problem built on themes of Visual Design, Social Value, Customer Design and Machine Learning}}
    \vspace*{-8pt}\item{\textbf{\href{https://github.com/Rochan-A/gym-dino/}{DinoEnv Gym Environment}}{: OpenAI Gym environment based on the Google Chrome Dino game}}
    \vspace*{-8pt}\item{\textbf{CineLog}{: An Intelligent and Adaptive framework to assist theaters predict sales \& footfall}}
    \vspace*{-8pt}\item{\textbf{Harmonize}{: Proof of Concept of a decentralized blockchain based Music Publishing \& Sharing Platform}}
    \vspace*{-8pt}\item{\textbf{\href{https://github.com/Rochan-A/oWatcher}{oWatcher}}{: Discord bot to display detailed in-game performance statistics of a players in Overwatch by Blizzard Entertainment}}
\resumeSubHeadingListEnd

%--------PROGRAMMING SKILLS------------%
\section{Technical Skills}
\resumeSubHeadingListStart
  \item{\textbf{Programming Languages}{: Python, C, R, Rust, C++, Golang, Java}
  \vspace*{-8pt}\item{\textbf{Libraries}{: OpenCV, Gym, PyTorch, NumPy, SciPy, Pandas, scikit-learn}}
  \vspace*{-8pt}\item{\textbf{Platforms}{: Nvidia DGX, AWS, GCP, Linux}}}
  \vspace*{-8pt}\item{\textbf{Tools}{: \LaTeX, Git}
\resumeSubHeadingListEnd

%-----------Awards-----------------
\section{Awards \& Service}
    {\small
    \begin{adjustwidth}{10pt}{}
        First Place, Smart India Hackathon -- 2020 \\
        Academic Scholarship, Mahindra \'{E}cole Centrale -- 2017-2018 \\
        President (Fmr. Vice-President), Mahindra \'{E}cole Centrale Computer Science Club -- 2019, 2018 \\
        Oral Presentation, Undergraduate Research Symposium, Mahindra \'{E}cole Centrale -- 2019 \\
        Finalist, Intel International Science and Engineering Fair (ISEF) -- 2017 \\
        Grand Award Winner \& Finalist, Intel Initiative for Research and Innovation in Science (IRIS) -- 2016 \\
        Finalist, Intel Initiative for Research and Innovation in Science (IRIS) -- 2015, 2012 \\
        Most Innovative Solution, Open European Championship FIRST Lego League -- 2014
    \end{adjustwidth}
    }

    % \resumeSubHeadingListStart
    %     \resumeSubItem{\textit{Third Place} - E-Summit Start-Up Sprint, Hyderabad, India, 2020}
    %         {Showcased a prototype of a distributed computing platform using mobile devices.}
    %     \resumeSubItem{\textit{First Place} - Dell Hack2Hire Hackathon, Hyderabad, India, 2019}
    %         {Proposed a multifaceted solution to the cart abandonment problem faced by e-commerce companies.}
    %     \resumeSubItem{\textit{Special Mention} - Pragyan Hackathon, Bangalore, India, 2018}
    %         {Showcased an Adaptive Movie Scheduling Framework for Multiplexes using RNN's, LSTM, Sentiment Analysis and Machine Learning}
    %   \resumeSubItem{\textit{First Place} - Brave Hackathon, Hyderabad, India, 2017}
    %         {Showcased a Decentralized Music Publishing and Sharing Platform built over 24 Hours}
    %   \resumeSubItem{\textit{Finalist} - Intel International Science and Engineering Fair (ISEF), Los Angeles, United States, 2017}
    %         {Showcased a research project titled - A Lossless Video Compression Technique Using Bayesian Networks and Entropy Coding}
    %   \resumeSubItem{\textit{Grand Award Winner \& Finalist} - Intel Initiative for Research and Innovation in Science (IRIS), Pune, India, 2016}
    %         {Showcased a research project titled - A Lossless Video Compression Technique Using Bayesian Networks and Entropy Coding}
    %   \resumeSubItem{\textit{Finalist} - Intel Initiative for Research and Innovation in Science (IRIS), 2015 \& 2012}
    %         {Showcased a research project on am Efficient Thermoelectric Refrigeration System; Showcased a solar cooker design that was inspired from the Parabolic and Box type designs}
    %   \resumeSubItem{\textit{Most Innovative Solution} - Open European Championship FIRST Lego League, Mannheim, Germany, 2014}
    %         {Showcased a research project on the Utilization of T2 Bacteriophages to fight Food Contamination}
    %   \resumeSubItem{\textit{Best Robot Design} - FIRST Lego League (FLL), Regionals, Bangalore, India, 2013}
    %         {Showcased a robot design that was highly efficient while still remaining simplistic}
    % \resumeSubHeadingListEnd

%-----------Positions of Responsibility-----------------
% \section{Positions of Responsibility}
%     \resumeSubHeadingListStart
%             \resumeSubheading
%                 {President}{Hyderabad, India}
%             {Enigma, Computer Science Club at Mahindra \'{E}cole Centrale}{2019 - 2020}
%             \resumeSubheading
%                 {Vice President}{Hyderabad, India}
%             {Enigma, Computer Science Club at Mahindra \'{E}cole Centrale}{2018 - 2019}
%             \resumeSubheading
%                 {Organizing Committee}{Hyderabad, India}
%             {MECHacks - 36 hour Hackathon at Mahindra \'{E}cole Centrale}{November 2018}
%         \resumeSubheading
%             {Organizing Committee}{Hyderabad, India}
%             {Mozilla Hackathon - Organized a 36 hour Hackathon with Mozilla Reps Community}{October 2018}
%         \resumeSubheading
%             {Student Member}{Intl.}
%             {ACM \& IEEE Student Member; Sigma Xi Member}{Since 2018}
%     \resumeSubHeadingListEnd

%-----------Conferences, Seminars & Workshops-----------------
\section{Seminars \& Workshops}
    \resumeSubHeadingListStart
        % \resumeSubheading
        % \resumeSubheading
        %     {IEEE Region 10 Symposium (IEEE TENSYMP)}{Kolkata, India}
        %     {Presented paper, IEEEXplore Digital Library: ISBN: 978-1-7281-0297-9}{May 2019}
        % \resumeSubheading
        %     {International Conference on Machine Learning and Data Science (ICMLDS)}{Hyderabad, India}
        %     {Attended two day conference held at Mahindra \'{E}cole Centrale}{Dec 2018}
%       \resumeSubheading
%           {MECHacks Hackathon}{Hyderabad, India}
%           {Organized a 36 hour Hackathon as a part of Aether, the annual fest at Mahindra Ecole Centrale}{November 2018}
        \resumeSubheading
            {Generative Art Workshop}{Hyderabad}
            {Conducted an introductory session on Generative Art as a part of the Computer Sci. Club}{November 2018}
        \resumeSubheading
            {Photo-Realistic Rendering Workshop}{Hyderabad}
            {Conducted a session on Photo-Realistic Rendering as a part of the Computer Sci. Club}{November 2018}
        \resumeSubheading
            {Python Workshop}{Hyderabad}
            {Conducted a week long Python Workshop for Freshers}{Sept 2018}
    \resumeSubHeadingListEnd

%-----------Relevant Courses-----------------
% \section{Relevant Courses}
%     \resumeSubHeadingListStart
%         \resumeSubItem{Core}{Operating Systems, Principles of Programming Languages, Database Management Systems, Microprocessors, Theory of Computation, Design \& Analysis of Algorithms, Digital Logic \& Design of Computer Architecture, Discrete Mathematics, Data Structures}
%         \resumeSubItem{Mathematics}{Partial Differential Equations, Numerical Methods, Probability, Random Processes \& Statistics, Linear Algebra, Calculus}
%     \resumeSubHeadingListEnd



%-----------Extra-Curricular Activities-----------------
% \section{Extra-Curricular Activities}
%     \begin{itemize}
%         \item \textbf{Football} : Member of the Mahindra Ecole Centrale College Football A Team, represent the college at Inter-College Meets.
%         \item \textbf{Swimming} : Member of the Mahindra Ecole Centrale College Swimming Team, represent the college at Inter-College Meets; Participated in the FINIS State level Sub Junior/Junior Championship event – Freestyle, Butterfly \& Individual Medley events; Represented Delhi Public School Bangalore East School Swim Team in a number of inter-school swimming competitions. Secured additional credits in CBSE Board.
%         \item \textbf{Running} : 1st Place in the 5 km ‘Spirit of Wipro’ marathon run (Open non-employee category); Timing: 21 mins
%         \item \textbf{Music} : Learnt playing Tabla (Indian percussion instrument). Cleared the Aadhya, Madya, Purna, 1st  year and 2nd year exams conducted by ‘Bangiya Sangeet Parishad’, Rabindra Bharati University, Kolkata.
%     \end{itemize}

\end{document}
